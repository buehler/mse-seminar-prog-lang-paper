\section{Introduction}

All modern programming languages have type systems.
They are either of a more dynamic nature - like JavaScript -
or statically typed like C-Sharp. Even if we talk about functional
languages like Haskell, they have a type system that helps
the developer to create programs.

But why are type systems helpful and how do they work
is not a trivial question to be answered. Consider
the following code statement:

\begin{lstlisting}[caption={Untyped language},captionpos=b]
    foo = "Hello World"
    bar = 42
    foo + bar //?
\end{lstlisting}

As humans, we immediately understand that this statement
is not going to terminate well - assuming we have an untyped
language. Strings and numbers are not
of the same type and cannot be added together. To determine
that this is not going to work, the computer needs to execute
the statements one by one and will encounter a wrong state.

A type system can prevent such errors and create a
human readable message when compiling such a program.
To understand a type system, it is necessary to use
maths to explain how we can check and derive types.
Such a mathematical system is the $\lambda$-Calculus.

The reader should have an understanding in programming
languages and a brief understanding of the untyped
$\lambda$-Calculus which is described in the first
chapters of \cite{pierce2002ProgLang}.

The following chapters will define some knowledge about
the untyped $\lambda$-Calculus and then introduce the
reader to the simply typed $\lambda$-Calculus.
