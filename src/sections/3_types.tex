\section{Types and Simple Types}
\label{sec:types}

\epigraph{Well-typed programs cannot ``go wrong.''}{Robin Milner (1978)}

To get one step closer towards the goal of having a
simply-typed $\lambda$-Calculus, we need to define
what a type is.

\subsection{Inference Rules}

The rest of the paper states some typed relations with
``inference rules''. As an example, consider the following
expression:

\begin{equation*}
    \tag{Inference Rule}
    \infer{
        C
    }{
        A & B
    }
\end{equation*}

Inference rules can be read as ``if this than that''. They act
similar to an \textit{if} statement without an else part of a programming language.
If A and B are valid, C follows. They do not give information about the
case when A or B is not valid.

A more complex example, as shown in \cref{eq:BoolInference}, is:

\begin{equation*}
    \infer{
        \text{if } t_1 \text{ then } t_2 \text{ else } t_3 \colon \texttt{T}
    }{
        t_1 \colon \texttt{Bool} & t_2 \colon \texttt{T} & t_3 \colon \texttt{T}
    }
\end{equation*}

This reads as: if the term ``$t_1$'' is of a boolean type and the terms ``$t_2$''
and ``$t_3$'' are of type \texttt{T},
then the resulting type of the expression \texttt{if $t_1$ then $t_2$ else $t_3$} is
of the type \texttt{T}. All elements above the line the antecedents that must be met
and below the line are the consequences.

\subsection{Types}

A type is a classification of a value or multiple values \cite{pierce2002ProgLang}.
We typically use mathematical terms to describe a type and the relation
of values to their types. When we say ``$x : T$'', we mean that
``x is of type T'', or in mathematical terms: ``$x \in T$'' \cite{pierce2002ProgLang}.
This relation makes it possible to determine the type of computation
that correlates with x and can be \textit{statically} analyzed.
For example:

\begin{equation*}
    \begin{cases}
        \text{if }   & x,y \in \mathbb{N}            \\
        \text{then } & f(x,y) = x + y \in \mathbb{N}
    \end{cases}
\end{equation*}

We can determine the resulting type of \textit{f} without
\textit{running} the function. Shown in TypeScript syntax this could mean:

\begin{minted}{TypeScript}
    const x: number = 1
    const y: number = 2
    const f = (x, y) => x + y
    f(x, y) // result is also of type number
\end{minted}

In TypeScript and JavaScript, a ``number'' is any numerical
value. Nevertheless, the type-checker can determine the return type
of the function based on the previous variables and their types.

With such a possibility, we can rule out stuck or meaningless programs.
To achieve this in a programming language we add the typing relation to
the grammar and define that terms and variables must have a type:

\begin{bnfgrammar}
    t ::= : terms$\colon$
    | x$\colon$T : variable
    | $\lambda$x$\colon$T.t : abstraction
    | \dots
\end{bnfgrammar}

To express the ``:T'' part in the grammar correctly, a new additional
grammar must be created to express types correctly:

\begin{bnfgrammar}
    T ::= : types$\colon$
    | Bool : type of booleans
    | Nat : type of natural numbers
\end{bnfgrammar}

And in addition to the syntactic definition, we need typing rules
that define the types of the arithmetic expression on top of
the rules that are given by the untyped $\lambda$-Calculus.

\begin{center}
    \texttt{true : Bool}
    
    \texttt{false : Bool}
    
    \texttt{0 : Nat}
\end{center}

\begin{equation*}
    \label{eq:BoolInference}
    \tag{If inference}
    \infer{
        \text{if } t_1 \text{ then } t_2 \text{ else } t_3 \colon \texttt{T}
    }{
        t_1 \colon \texttt{Bool} & t_2 \colon \texttt{T} & t_3 \colon \texttt{T}
    }
\end{equation*}

\[\infer{ \text{succ } t_1 \colon \texttt{Nat} }{ t_1 \colon \texttt{Nat} }\]

\[\infer{ \text{pred } t_1 \colon \texttt{Nat} }{ t_1 \colon \texttt{Nat} }\]

\[\infer{ \text{iszero } t_1 \colon \texttt{Bool} }{ t_1 \colon \texttt{Nat} }\]

This means we allocate \texttt{true} and \texttt{false} to the \texttt{Bool} type.
Then we assign \texttt{0} to the \texttt{Nat} type. 
The derivation rules state that all \texttt{succ} (successors) and \texttt{pred} (predecessors) are of the
\texttt{Nat} type. The rules for the predecessor and successor define that the result
of the applied \texttt{succ} or \texttt{pred} function must be of type \texttt{Nat}. The rule for the condition
defines that the input for the ``\texttt{if}'' must be a boolean value and the result is
of type \texttt{T}. Both branches of the condition must have the same type.

\subsubsection{Type-Safety}

The given typing rules give our type system a pretty important property:
\textit{safety} (or \textit{soundness}) \cite{pierce2002ProgLang}.
The normalization property, which defines that for every term, there exists a rewrite rule that will
yield an irreducible term \cite{pierce2002ProgLang,baader_nipkow_1998}.
This property eliminates the Turing completeness in the untyped $\lambda$-Calculus, 
we can guarantee that our programs
that compile successfully with this type system will terminate. We can call
such a program well-typed (i.e., it compiles according to the given typing rules \cite{cardelliTypeSystems}).

This \textit{safety} is defined by two theorems \cite{pierce2002ProgLang}:
\begin{itemize}
    \item \textit{Progress}: Well-typed terms are not stuck. They can take
          a step in the evaluation rules or are a value.
    \item \textit{Preservation}: A well-typed term that takes a step in the evaluation rules
          will yield a result that is also well-typed.
\end{itemize}

\subsection{Intermediate Result}

With the given syntax, evaluation rules and type definitions, we would
have a type system that could successfully compile the following
lines of code (the syntax is inspired by TypeScript for a clear
reference to a computer program):

\begin{minted}{TypeScript}
    const tr: boolean = true
    const x: number = 42

    if (tr) {
        x
    } else {
        x + 1
    }
\end{minted}

\subsection{Simple Types}

Alonzo Church defined the theory of simple types
\cite{churchLogic}. In combination with the examples and
statements of Benjamin C. Pierce \cite{pierce2002ProgLang}, there
exists a definition of a simple type. Simple types are a first
approach towards a typesafe environment for developers. They
contain ``base types'' (or ``value types'') like \texttt{Bool} and
\texttt{Nat} (natural numbers) as well as ``function types''.
A language that only contains simple types and adheres to
the principles of the previous sections is guaranteed to terminate.

When we apply ``simple types'' to the purely untyped
$\lambda$-Calculus, the result is the ``pure simply-typed $\lambda$-Calculus''.
This could count as a programming language since we can perform
all the basic operations a computation needs. We are also able to analyze
the code and calculate the types needed for functions and variables
and therefore can categorize our programs into meaningful and
meaningless ones.

Many imperative programming languages are simply-typed since they do not
know the (classic) concepts of ``classes'' and ``polymorphism''. Examples for such languages
are:

\begin{itemize}
    \item BASIC
    \item Lua
    \item Go
\end{itemize}

BASIC\footnote{\url{https://en.wikipedia.org/wiki/BASIC}} for example, is an
old language that does not contain any constructs like classes or
functions\footnote{Later extensions of BASIC enabled object-orientation though.}.

\begin{minted}{BASIC}
    10 LET X=5
    20 FOR I=1 TO N
    30 PRINT "Hello, World! I="; $I
    40 NEXT I
\end{minted}

The program above prints five times ``Hello, World! I='' and the actual
number of I.

The other two examples are modern programming languages which work without
classic polymorphism and object-oriented inheritance while both contain
some mechanism to ``simulate'' polymorphism. Lua\footnote{\url{https://www.lua.org/}}
handles polymorphism via first-class functions and tables. Go\footnote{\url{https://golang.org/}}
on the other side, only knows ``structs'' as data types. Structs are similar to
``Records'' explained in \cref{subsec:Records}. With the help of \textit{interfaces}
Go is able to provide \textit{runtime polymorphism}.

\subsubsection{Base Types}

Base types represent unstructured values in a programming language \cite{pierce2002ProgLang}.
An incomplete list of such base types we will encounter is:

\begin{itemize}
    \item Numbers (\texttt{Integers} and \texttt{Float})
    \item Booleans
    \item Strings (list of \texttt{Characters})
\end{itemize}

Since base types are unspectacular and are used to calculate other
types, the literature often substitutes them with a letter for all
\textit{unknown} base types \cite{pierce2002ProgLang}. Often,
constructs like $\mathcal{A}$ or other letters are seen. In this
paper, we will establish the following syntactic form
from Benjamin C. Pierce \cite{pierce2002ProgLang}:

\begin{bnfgrammar}
    T ::= : types$\colon$
    | A : base type
\end{bnfgrammar}

The given \texttt{A} "type-class" replaces the previous stated
``\texttt{Nat}'' and ``\texttt{Bool}'' types. \texttt{A} now contains
all base types.

\subsubsection{Function Types}

From a theoretical perspective, this language is quite interesting, but to be used
as a programming language, it lacks some needed features. For example, we need the
possibility to apply a function to some input to generate some output. Otherwise,
this programming language will be quite boring. To add functions to our typing
syntax, we add the following line:

\begin{bnfgrammar}
    T ::= : types$\colon$
    | \dots
    | T$\rightarrow$T : type of functions
\end{bnfgrammar}

A \textit{type environment} ($\Gamma$) is introduced in the following derivation rules.
This environment (or sometimes called \textit{type context}) is
a mathematical set with variables and their types \cite{pierce2002ProgLang}.
When our type-checker starts, the starting environment equals
``$\varnothing$'' (the empty set). With each evaluation step, this set will grow
and contain the specified values.

Now we have the typing syntax for functions, but we need some additional typing
rules to ensure the types of functions can be calculated statically:

\begin{equation*}
    \tag{Abstraction}
    \infer{
        \Gamma \vdash \lambda x \colon \texttt{T}_1 . t_2 \colon \texttt{T}_1 \rightarrow \texttt{T}_2
    }{
        \Gamma,x \colon \texttt{T}_1 \vdash t_2 \colon \texttt{T}_2
    }
\end{equation*}

This typing rule for the general abstraction evaluation rule
of the $\lambda$-Calculus adds
a premise to our system that translates to ``if $x$ is of type
$T_1$, bound in the context $\Gamma$ and the term $t_2$ yields the type $T_2$,
then the abstraction $\lambda x \colon T_1 . t_2$ is bound to the type $T_1 \rightarrow T_2$''.

Furthermore, an additional typing rule for variables and applications
are needed:

\begin{equation*}
    \tag{Variable}
    \infer{
        \Gamma \vdash x \colon \texttt{T}
    }{
        x \colon \texttt{T} \in \Gamma
    }
\end{equation*}

The type that is assumed for x is in the set of $\Gamma$. The inference
rule above states that ``if the type \texttt{T} of $x$ is an element of
$\Gamma$, then bind the term $x$ to that type in the context $\Gamma$''.

\begin{equation*}
    \tag{Application}
    \infer{
        \Gamma \vdash t_1\,t_2 \colon \texttt{T}_{12}
    }{
        \Gamma \vdash t_1 \colon \texttt{T}_{11} \rightarrow \texttt{T}_{12}
        &
        \Gamma \vdash t_2 \colon \texttt{T}_{11}
    }
\end{equation*}

If $t_1$ is a function that takes $T_{11}$, returns
$T_{12}$ and is bound in the context $\Gamma$;
and the term $t_2$ is a value of type $T_{12}$ in the context $\Gamma$, then
the result of the application of $t_1$ to $t_2$ will
be of type $T_{12}$ and is bound into the context.
\\[12pt]
Translated into a programming language:

\begin{minted}{TypeScript}
    // number (Variable)
    const x = 1337

    // number => string (Abstraction)
    const f = nr => nr.toString()

    // number that becomes a string (Application)
    const r = f(x)
\end{minted}

In general, translated to ``developer speech'', the variable
remains the same, an abstraction is a function and the application
is the execution of a function to get a result.
