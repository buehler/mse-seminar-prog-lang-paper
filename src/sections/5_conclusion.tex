\section{Conclusion}

With the application of simple types to the $\lambda$-Calculus
and the addition of ``simple extensions'', the result of this paper
is a form of the ``simply-typed $\lambda$-Calculus'' with extensions to
the language and the type system to make a practical language. It shows
the additions needed and the steps taken to add a type-system to an
untyped language. It gives some insight into the very complex topic
of type systems and how they work and provides explanations for the
various mathematical formulas and derivation trees. It should be
noted that the result is not a fully-fledged programming language since
subtyping and polymorphism is still missing which is a big part of
modern languages. The simply-typed $\lambda$-Calculus however, is
a good example for type-science since it is not too complex and
can be extended quite easily.

Below all described additions and extensions will be shown in
TypeScript syntax as well as Haskell syntax to give a practical
example of the created type-system. Not all elements are
translateable 1:1.

\subsection*{TypeScript}

\begin{minted}{TypeScript}
// Base Types
const foo: string = 'bar';
const nr: number = 42;

// Function Types
const f = () => console.log('');
const fx = x => x * x;

// Unit Type
function f(): void {}

// Sequencing
console.log(42);
console.log(1337);

// Ascription
const foo = 'something' as string;

// Let Bindings
let foo = (x, y) => x+y; // let
const result = foo(1,2) // "in" ...

// Pairs and Tuples
const foo = [42, 'something', true];

// Records
const foo = {name: 'Max', lastname: 'Muster'};

// Sums and Variants
type MyVariant = string | number; // Union Type

// General recursion
const factorial = (n) => n <= 1
    ? 1
    : n * factorial(n-1)

// Lists
const arr: string[] = ['one', 'two', 'three'];

// References
let variable = 1;
const f = () => variable = variable + 1;
f();

// Exceptions
throw 'Something went wrong';
\end{minted}

\subsection*{Haskell}

\begin{minted}{Haskell}
-- Base Types
foo = "string"
nr = 42

-- Function Types
f x y = x + y

-- Unit Type
-- Zero element tuple
empty = ()

-- Sequencing
main = do
    a <- something
    b <- something
    return 0

-- Ascription
type MyType = Double * [Char]

-- Let Bindings
f x y =
    let calc a b = a + b
    in calc x y

-- Pairs and Tuples
pair = (1, 2)
tuple = (1, 2, 3)

-- Records
data Person = Person
    { name :: String
    , lastname :: String
    }

-- Sums and Variants
data Maybe a = Just a | Nothing
data Weekdays = Monday
    | Tuesday
    | Wednesday
    | Thursday
    | Friday
    | Saturday
    | Sunday

-- General recursion
factorial 0 = 1
factorial 1 = 1
factorial n = n * (factorial n-1)

-- Lists
list = 1 :: 2 :: 3 :: []

-- References
-- Doable with Monads
-- for example with the State-Monad

// Exceptions
error "Something went wrong"
\end{minted}
