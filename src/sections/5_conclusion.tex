\section{Conclusion}

With this paper, we showed the usefulness of type-systems
and their application on a ``toy language'', the $\lambda$-Calculus.

The reader was introduced into the general topic in \cref{sec:intro}.
In \cref{sec:calculus}, the paper explains the ways of expressing
a program and how this is done in the untyped $\lambda$-Calculus.

In \cref{sec:types} the paper gives an overview over
the topic of ``types'', what they are, their use and
how they are constructed. Also, a general understanding about
``inference rules'' is given to the reader. The section
concludes with the explanation of ``simple types''.

\cref{sec:lambdaTypesExtensions} then describes the
pure simply-typed $\lambda$-Calculus and what simple extensions
are. The paper gives a list with the extensions from the work
of Benjamin C. Pierce in \cite{pierce2002ProgLang} and a brief
explanation. Furthermore, three interesting extensions are
given a deeper look into how they are added to the language.

With the application of simple types to the $\lambda$-Calculus
and the addition of ``simple extensions'', the result is a form of the
``simply-typed $\lambda$-Calculus'' with extensions to
the language and the type system to make a practical language.

It should be noted that the pure simply-typed $\lambda$-Calculus
with its extensions is not a fully-fledged programming language since
subtyping and polymorphism are still missing which is a big part of
modern languages. The simply-typed $\lambda$-Calculus, however, is
a good example for type-science since it is not too complex and
can be extended quite easily.

Further study into the topic may now include complex topics like
``polymorphism''.
