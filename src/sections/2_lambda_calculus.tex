\section{Lambda Calculus}

A computer program can be described in multiple
ways. A. M. Turing defined the turing machine
in \cite{aturingMachine} in the year 1937. This
machine was an abstract model of computation.
In contrast to a "machine", the mathematical model
of a computer program was defined by Alonzo Church
\cite{churchLogic}. It uses mathematical logic
to run calculations. All elements can be described
in mathematical terms.

The untyped $\lambda$-Calculus is turing complete,
which means it can compute \textit{any} program and
therefore can run infinitely. This is a contrast to
the simply typed calculus, which limits the executable
terms in the way that they never reach an erroneous state.
In an untyped $\lambda$ system, it is possible to search
for the successor of "true", which
requires the argument to be a number and therefore
results in a stuck state.

Since it is not desirable for computer programs to run to infinity,
the mathematical system of types \cite{churchLogic} was defined.

% Very briefly describe the untyped lambda calculus
% with reference to the paper of Simon Gubler
% (which writes the paper for the untyped calculus)

% The brief description should provide an overview of
% terms and elements so that the reader can advance
% to the transition to the typed system.

% \begin{enumerate}
%     \item simple syntax example
%     \item simple rules
%           \begin{itemize}
%               \item variables
%               \item abstractions
%               \item applications
%           \end{itemize}
%     \item simple typescript example \cite{lambdaCalcInTS}
% \end{enumerate}
