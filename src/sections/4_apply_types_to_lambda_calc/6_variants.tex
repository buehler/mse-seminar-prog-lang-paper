\subsection{Sums and Variants}
\label{subsec:variants}

Many programs need to tackle variants. This means
that we can sum together multiple shapes of a type into a summary
type. Such variants are \textit{algebraic data types} and are often used in functional
languages with pattern matching. One can compare them vaguely to \textit{Enums} of
object-oriented languages like C\#.
This paper will use the generalized definition
of the variant to describe the principle. So instead of a sum type \texttt{T + T}
we use the labeled variant type $\langle l_1 \colon \texttt{T}_1, l_2 \colon \texttt{T}_2 \rangle$.
The sum type could be compared to Haskells \textit{Either} type, which can be
either type ``a'' or ``b''.

Addition to the syntax \cite{pierce2002ProgLang}:
\begin{bnfgrammar}
    t ::= : terms$\colon$
    | \dots
    | <l=t> as T : tagging
    | case t of <$l_i=x_i$> $\Rightarrow t_i^{i \in 1..n}$ : case
\end{bnfgrammar}\leavevmode\newline
This syntax allows generalized labeled variants of types.

\begin{bnfgrammar}
    T ::= : types$\colon$
    | \dots
    | <$l_i \colon T_i^{i \in 1..n}$> : type of variants
\end{bnfgrammar}\leavevmode\newline

Addition to the evaluation rules \cite{pierce2002ProgLang}:
\begin{equation*}
    \tag{Case variant}
    \begin{split}
        \texttt{case} (\text{<}l_j=v_j\text{>} \texttt{ as } \texttt{T}) \texttt{ of } \text{<}l_i=x_i\text{>} \Rightarrow t_i^{i \in 1..n} \\
        \rightarrow [x_j \mapsto v_j]t_j
    \end{split}
\end{equation*}
``Check the variant with a \texttt{case variant} syntax and return
the given term to the righthand side of the arrow. Replace the
x variable in the term with the ascribed type''.

\begin{equation*}
    \tag{Case}
    \infer{
        \begin{split}
            \texttt{case } t_0 \texttt{ of } \text{<}l_i=x_i\text{>} \Rightarrow t_i^{i \in 1..n} \\
            \rightarrow \texttt{case } t'_0 \texttt{ of } \text{<}l_i=x_i\text{>} \Rightarrow t_i^{i \in 1..n}
        \end{split}
    }{
        t_0 \rightarrow t'_0
    }
\end{equation*}
``If there is a step from $t_0$ to $t'_0$, evaluate the term
in the case clause before applying any mappings''.

\begin{equation*}
    \tag{Variant}
    \infer{
        \text{<}l_i=t_i\text{>} \texttt{ as T } \rightarrow \text{<}l_i=t'_i\text{>} \texttt{ as T }
    }{
        t_i \rightarrow t'_i
    }
\end{equation*}
``If there is a step from $t_i$ to $t'_i$, evaluate the term
in the variant''.

~\\
Addition to the typing rules \cite{pierce2002ProgLang}:
\begin{equation*}
    \tag{Variant}
    \infer{
    \Gamma \vdash \text{<}l_j=t_j\text{>} \texttt{ as } \text{<}l_i \colon T_i^{i \in 1..n}\text{>} \colon \text{<}l_i \colon T_i^{i \in 1..n}\text{>}
    }{
    \Gamma \vdash t_j \colon \texttt{T}_j
    }
\end{equation*}
``When the type of $t_j$ is in the type environment, add the labeled
variant types in the environment as well with their corresponding labels and
term types''.

\begin{equation*}
    \tag{Case}
    \infer{
    \Gamma \vdash \texttt{case } t_0 \texttt{ of } \text{<}l_i=x_i\text{>} \Rightarrow t_i^{i \in 1..n} \colon \texttt{T}
    }{
    \begin{split}
        \Gamma \vdash t_0 \colon \text{<}l_i \colon \texttt{T}_i^{i \in 1..n}\text{>} \\
        \text{for each } i \text{    } \Gamma, x_i \colon \texttt{T}_i \vdash t_i \colon T
    \end{split}
    }
\end{equation*}
``If $t_0$ is a variant with $i$ labels (and therefore variants), the `case of' syntax
will return the specific type of the variant instead of the summary type''.

With the variant types in place, our language could now type-check and interpret
the following lines of code (given in the Haskell syntax for readability):

\begin{minted}{Haskell}
    data StringOrInt = MyString String | MyInt Int
    getStringValue :: StringOrInt -> String
    getStringValue value = case value of
        MyString s -> s
        MyInt i -> show i
\end{minted}

Variants are often used to represent variable return values.
One can think of the ``Option'' type in F\# or the ``Maybe'' type
of Haskell. Both define two variants of a result, namely ``Some'' (``Just'')
or ``None'' (``Nothing''). A computation that may have a none result can return
this type constructor of the variant and can signal an empty or faulty result.
A typical use-case could be number parsing. When one wants to parse the string
\texttt{"12"} into a number, the result in Haskell could be \texttt{"Just 12"}, but
on the other hand parsing \texttt{"12i"} would result in a \texttt{"Nothing"}.
