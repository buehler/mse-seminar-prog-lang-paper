\subsection{Pairs and Tuples}

Until now, the additions were minor and added some syntactic sugar to the language of
the simple typed $\lambda$-Calculus. The following simple extensions will enrich
the language with features that are often found in programming languages.

Pairs - and their more general counterpart Tuples - are a construct to group
values and terms together. Pairs are product types of exactly two values
and therefore have slightly different evaluation rules. Tuples are the general way
of pairs and therefore only the rules of tuples will be explained since they include
the rules of pairs as well.

\subsubsection{Addition to the syntax \cite{pierce2002ProgLang}}
\begin{bnfgrammar}
    t ::= : terms$\colon$
    | \dots
    | $\{t_i^{i \in 1..n}\}$ : tuple
    | t.i : projection
\end{bnfgrammar}\leavevmode\newline
$\{t_i^{i \in 1..n}\}$ means that for example with $i=3$ we have a tuple
of three elements. A tuple with $i=2$ would be a pair.
$\{t_i^{i \in 1..n}\} \text{ with } i=3 \mapsto \{t_1, t_2, t_3\}$.
The projection is needed to access the elements in a tuple at the given index.

\begin{bnfgrammar}
    v ::= : values$\colon$
    | \dots
    | $\{v_i^{i \in 1..n}\}$ : tuple value
\end{bnfgrammar}\leavevmode\newline

\begin{bnfgrammar}
    T ::= : types$\colon$
    | \dots
    | $\{T_i^{i \in 1..n}\}$ : tuple type
\end{bnfgrammar}\leavevmode\newline

\subsubsection{Addition to the evaluation rules \cite{pierce2002ProgLang}}
\begin{equation*}
    \tag{Tuple projection}
    \{v_i^{i \in 1..n}\} . j \rightarrow v_j
\end{equation*}
``When the projection with index $j$ is applied to a tuple with
$i$ values, return the value at index $j$''.

\begin{equation*}
    \tag{Projection}
    \infer{
        t_1 . i \rightarrow t'_1 . i
    }{
        t_1 \rightarrow t'_1
    }
\end{equation*}
``If there is a step from $t_1$ to $t'_1$, evaluate the step
in the syntax before executing the projection''.

\begin{equation*}
    \tag{Tuple}
    \infer{
    \{v_i^{i \in 1..j-1}, t_j, t_k^{k \in j+1..n}\} \rightarrow \{v_i^{i \in 1..j-1}, t'_j, t_k^{k \in j+1..n}\}
    }{
    t_j \rightarrow t'_j
    }
\end{equation*}
``If there is a step from $t_j$ to $t'_j$, evaluate the leftmost
term $t_j$ to $t'_j$ that is not a value''. This forces the tuple to
be fully evaluated before any projections can be executed on the tuple.
Also, it enforces the evaluation direction for the tuple from left to write.
In other terms: $\{t_1, t_2\} \mapsto \{v_1, t_2\} \mapsto \{v_1, v_2\}$.

\subsubsection{Addition to the typing rules \cite{pierce2002ProgLang}}
\begin{equation*}
    \tag{Tuple}
    \infer{
    \Gamma \vdash \{t_i^{i \in 1..n}\} \colon \{\texttt{T}_i^{i \in 1..n}\}
    }{
    \text{for each } i & \Gamma \vdash t_i \colon \texttt{T}_i
    }
\end{equation*}
``For each element in the tuple with index $i$, we calculate the type
and add the whole tuple to the typing context $\Gamma$ in the form $\{T_1, T_2, \dots\}$''.

\begin{equation*}
    \tag{Projection}
    \infer{
    \Gamma \vdash t_i . j \colon \texttt{T}_j
    }{
    \Gamma \vdash t_1 \colon \{\texttt{T}_i^{i \in 1..n}\}
    }
\end{equation*}
``If the term $t_1$ is a tuple type with $i$ entries, the projection
$t_1 . j$ will yield an element of the type $\texttt{T}_j$''.
