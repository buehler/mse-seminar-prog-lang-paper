\subsection{Unit Type}

The \texttt{unit} type represents a useful type often found in functional
programming languages like Haskell or F\#. It is used to ``throw away'' a
computation result and combine multiple computations together \cite{pierce2002ProgLang}.
In Haskell, this can be used in the \texttt{main} function to glue
several functions together that contain side effects. It can be viewed as
the ``\texttt{void}'' type in C\# or Java \cite{pierce2002ProgLang}.

\subsubsection{Addition to the syntax \cite{pierce2002ProgLang}}
\begin{bnfgrammar}
    t ::= : terms$\colon$
    | \dots
    | unit : constant unit
\end{bnfgrammar}\leavevmode\newline

\begin{bnfgrammar}
    v ::= : values$\colon$
    | \dots
    | unit : constant unit
\end{bnfgrammar}\leavevmode\newline

\begin{bnfgrammar}
    T ::= : types$\colon$
    | \dots
    | Unit : unit type
\end{bnfgrammar}\leavevmode\newline

\subsubsection{Addition to the typing rules \cite{pierce2002ProgLang}}
\begin{equation*}
    \tag{Unit}
    \Gamma \vdash \texttt{unit} \colon \texttt{Unit}
\end{equation*}\leavevmode\newline

\subsubsection{Added derived form \cite{pierce2002ProgLang}}
\begin{equation*}
    t_1 ; t_2 \quad \defeq \quad (\lambda x \colon \texttt{Unit} . t_2) t_1 \text{ where } x \notin FV(t_2)
\end{equation*}
``The function is applied to the term $t_1$ where the input variable $x$ is not
part of the ``free variables''\footnote{Free variables are not bound variables in the term.}
(FV) of the term $t_2$''.

\subsubsection{Addition to the evaluation rules \cite{pierce2002ProgLang}}
\begin{equation*}
    \tag{Sequence}
    \infer{
        t_1 ; t_2 \rightarrow t'_1 ; t_2
    }{
        t_1 \rightarrow t'_1
    }
\end{equation*}
``If there is a sequence (noted by `;') and there is a step from $t_1$ to
$t'_1$, evaluate the term $t_1$''.

\begin{equation*}
    \tag{Sequence Next}
    \texttt{unit} ; t_2 \rightarrow t_2
\end{equation*}
``If the left-hand side of a sequence is reduced to a \texttt{unit} value,
return the result of $t_2$''.

\subsubsection{Addition to the typing rules \cite{pierce2002ProgLang}}
\begin{equation*}
    \tag{Sequence}
    \infer{
        \Gamma \vdash t_1 ; t_2 \colon \texttt{T}_2
    }{
        \Gamma \vdash t_1 \colon \texttt{Unit} & \Gamma \vdash t_2 \colon \texttt{T}_2
    }
\end{equation*}
``If $t_1$ is of type \texttt{Unit} and $t_2$ has type $\texttt{T}_2$, then
the resulting type of the sequence will be $\texttt{T}_2$''.
