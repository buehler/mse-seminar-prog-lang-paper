\subsection{Summary}

With those possibilities, a variety of concepts can be created.
While there exists the concept of ``exceptions'' for example, it does
not support subtyping of any kind. Polymorphism is part of a higher
version of a type system, for example System F.

\todo{reference to marcs paper about systemF}

The given base of the ``pure simple typed $\lambda$-Calculus'' and
the simple extensions above now give us a language that can successfully compile
and compute the following lines of (TypeScript-ish) code.

\todo{Code}

Since the $\lambda$-Calculus is the mathematical foundation of several functional
programming languages, the features (without polymorphism) can be translated
into them. Let us review the stated features in a pure functional language
like Haskell:

\todo{Code}
