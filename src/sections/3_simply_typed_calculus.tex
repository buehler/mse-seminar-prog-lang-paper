\section{Simply Typed Lambda-Calculus}

The mathematical abstraction of the $\lambda$-Calculus
is a topic that is not easy to understand if encountered
for the first time. The purpose of the present paper is to
describe the nature of the typed calculus with examples from
programming languages so that a developer can understand the
concepts.

% Short introduction sentence here.
% the following sub sections should describe
% the principles of the simply typed lambda calculus.

% \begin{itemize}
%     \item What are types
%     \item What can we do with them
%     \item further elements like base types
%     \item where is it used (lambda calculus)
% \end{itemize}

% each section should explain the topic in both ways.
% first the "math" way to explain what the topic is and
% then, the developer language version.

% The goal should be, that a developer can read those sections
% and can interpret what the simply typed lambda calculus is. Explained
% in terms and examples that a normal experienced developer should understand.

\subsection{Key Differences}

Briefly describe the key differences between
the untyped and the simply typed calculus.

further sections will specify those differences.

\subsection{Types}

Describe what types are and what types mean.
In terms of mathematics and programming languages.

Why they are needed for certain tasks.

\subsection{Types in the Calculus}

Describe the different types and their context.
What are function types and how are they used.
What are the properties of typing.

Ref to Curry/Church

typescript example of church notation with typings.

\subsection{Extending types}

\begin{itemize}
    \item What hides behind "base types"
    \item the unit type (void)
    \item other typing constructs for the simple calculus
\end{itemize}

\subsection{References}

\begin{itemize}
    \item What is a reference type
\end{itemize}
\subsection{Error states}

\begin{itemize}
    \item What exactly is an exception
    \item How to throw / handle them
\end{itemize}

