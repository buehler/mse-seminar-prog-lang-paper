\section{$\lambda$-Calculus, Simple Types and Extensions}
\label{sec:lambdaTypesExtensions}

To make the language of the $\lambda$-Calculus more useful, however, we can extend
our simple types with ``simple extensions'' which do not include any
form of polymorphism. Those extensions make our language and the type
checker more useful and able to perform operations.

``Simple types'' give a language the means to statically analyze the program
and determine if a program is ``useful'' or ``useless'' and therefore, if a
program is able to terminate. But only with those simple types, a programming
language would be tedious to use and many lines of code need to be written
for a meaningful program. Because of this matter, it is possible to define
extensions to the simple type system to let the simple typed $\lambda$-Calculus
language be more practical. The following sections will explain the
pure simply-typed $\lambda$-Calculus, an overview over the extensions and
three extensions in particular.

\subsection{Pure simply-typed $\lambda$-Calculus}

For the further reading, the rules of the ``pure simply-typed $\lambda$-Calculus''
are stated again \cite{pierce2002ProgLang}.

\subsubsection{Syntax}
\begin{bnfgrammar}
    t ::= : terms$\colon$
    | x : variable
    | $\lambda$x$\colon$T . t : abstraction
    | t t : application
\end{bnfgrammar}\leavevmode\newline

\begin{bnfgrammar}
    v ::= : values$\colon$
    | $\lambda$x$\colon$T . t : abstraction value
\end{bnfgrammar}\leavevmode\newline

\begin{bnfgrammar}
    T ::= : types$\colon$
    | T$\rightarrow$T : type of functions
\end{bnfgrammar}\leavevmode\newline

\begin{bnfgrammar}
    $\Gamma$ ::= : contexts$\colon$
    | $\varnothing$ : empty context
    | $\Gamma$, x$\colon$T : term variable binding
\end{bnfgrammar}\leavevmode\newline

\subsubsection{Evaluation}
\begin{equation*}
    \tag{Application 1}
    \infer{
        t_1\,t_2 \rightarrow t'_1\,t_2
    }{
        t_1 \rightarrow t'_1
    }
\end{equation*}
If there exists an evaluation step from $t_1$ to $t'_1$, take this step
prior to the evaluation step of $t_2$. Since there are no other applicable rules,
this forces the program to first evaluate all terms of $t_1$ until other rules become
applicable.

\begin{equation*}
    \tag{Application 2}
    \infer{
        v_1\,t_2 \rightarrow v_1\,t'_2
    }{
        t_2 \rightarrow t'_2
    }
\end{equation*}
If there exists an evaluation step from $t_2$ to $t'_2$ and the left
side of the application is already a value, take this step. This defines
that when the left-hand side of the application is reduced to a value, evaluate
the right-hand side.

\begin{equation*}
    \tag{Application Abstraction}
    (\lambda x \colon T_{11} . t_{12}) v_2 \rightarrow [x \mapsto v_2] t_{12}
\end{equation*}
Replace the variable $x$ with the value $v_2$ in the term $t_{12}$. This represents
the effective computation or application of the value to a term.

\subsubsection{Typing}
\begin{equation*}
    \tag{Variable}
    \infer{
        \Gamma \vdash x \colon \texttt{T}
    }{
        x \colon \texttt{T} \in \Gamma
    }
\end{equation*}

\begin{equation*}
    \tag{Application}
    \infer{
        \Gamma \vdash t_1\,t_2 \colon \texttt{T}_{12}
    }{
        \Gamma \vdash t_1 \colon \texttt{T}_{11} \rightarrow \texttt{T}_{12}
        &
        \Gamma \vdash t_2 \colon \texttt{T}_{11}
    }
\end{equation*}

\begin{equation*}
    \tag{Abstraction}
    \infer{
        \Gamma \vdash \lambda x \colon \texttt{T}_1 . t_2 \colon \texttt{T}_1 \rightarrow \texttt{T}_2
    }{
        \Gamma,x \colon \texttt{T}_1 \vdash t_2 \colon \texttt{T}_2
    }
\end{equation*}

This in combination with base and function types will be the
cornerstone of the ``simple extensions'' which we will see in the
next sections. Those sections will introduce new elements to the given
categories (``syntax'', ``evaluation'', ``typing'').

\subsection{Simple Extensions}

The work of Benjamin C. Pierce does list numerous extensions to the
pure simply-typed $\lambda$-Calculus. In later sections, ``Let Bindings'',
``Sums and Variants'' as well as ``Records'' are explained. The provided
extensions in the book are briefly listed below \cite{pierce2002ProgLang}:

\begin{itemize}
    \item \textit{Unit}: The \texttt{unit} type represents
          a useful type often found in functional
          programming languages like Haskell or F\#.
          It is used to ``throw away'' a computation result
          and combine multiple computations together \cite{pierce2002ProgLang}.
    \item \textit{Ascription}: A handy tool to substitute long type names with
          short ones. It is often used for ``documentation'' purposes \cite{pierce2002ProgLang}.
    \item \textit{Let Bindings}: Let bindings are a tool to
          avoid repetition in complex expressions \cite{pierce2002ProgLang}. They are explained in \cref{subsec:let}.
    \item \textit{Pairs and Tuples}: ``Pairs'' - and their more general counterpart
          ``Tuples'' - are a construct to group values and terms together.
          Pairs are product types of exactly two values
          and therefore have slightly different evaluation rules. Tuples are the general way
          of pairs \cite{pierce2002ProgLang}.
    \item \textit{Records}: ``Records'' provide a way to label the entries
          of a tuple and create a possibility to
          semantically group terms together \cite{pierce2002ProgLang}. They are explained in \cref{subsec:Records}.
    \item \textit{Variants}: Variants are \textit{algebraic data types} and are often used in functional
          languages for pattern matching. One can compare them vaguely to \textit{Enums} of
          object-oriented languages like C\#. They enable the language to sum together multiple
          shapes of a type into a summary type \cite{pierce2002ProgLang}. Find the sum
          and variant types explained in \cref{subsec:variants}.
    \item \textit{General Recursion}: The general recursion enables the language to
          ``iterate'' through a list and perform recursive computations. It is
          enabled via a language primitive (``\texttt{fix}'') as explained
          in the work by Pierce \cite{pierce2002ProgLang}.
    \item \textit{Lists}: A list is a practical and useful type constructor
          that describes a finite set of elements which are fetched from the
          type of the list. \cite{pierce2002ProgLang}
    \item \textit{References}: A mechanism to assign a value to a variable
          which can be changed later on. A reference enables a language to
          perform ``computational effects'' \cite{pierce2002ProgLang}.
    \item \textit{Exceptions}: Exceptions are a way of signaling that
          a function in question is not able to carry out a given task. This signal
          can be used for error handling at runtime \cite{pierce2002ProgLang}.
\end{itemize}

As seen, the given extension do create more meaning to the type system
and the language of the simply-typed $\lambda$-Calculus. One should
keep in mind that it is a ``toy language'' to study type systems and not
a practical programming language.

\subsection{Let Bindings}
\label{subsec:let}

Let bindings enable the language to avoid repetition for complex
terms and expressions. One can substitute a more complex term
(e.g., a calculation of some sort) with one ``name''.
They are found in Haskell as well:

\begin{minted}{Haskell}
    add x y =
        let result = x + y
        in
            result
\end{minted}

Let bindings are essentially a substitution for another term or
variable.

\subsubsection{Addition to the syntax \cite{pierce2002ProgLang}}
\begin{bnfgrammar}
    t ::= : terms$\colon$
    | \dots
    | let x = t in t : let binding
\end{bnfgrammar}\leavevmode\newline

\subsubsection{Addition to the evaluation rules \cite{pierce2002ProgLang}}
\begin{equation*}
    \tag{Let-Bind Value}
    \texttt{let } x = v_1 \texttt{ in } t_2 \rightarrow [x \mapsto v_1] t_2
\end{equation*}
In the given abstraction $t_2$, replace all occurencies of $x$ with $v_1$.

\begin{equation*}
    \tag{Let}
    \infer{
        \texttt{let } x=t_1 \texttt{ in } t_2 \rightarrow \texttt{let } x=t'_1 \texttt{ in } t_2
    }{
        t_1 \rightarrow t'_1
    }
\end{equation*}
If there is a step from $t_1$ to $t'_1$, evaluate the step
in the syntax before evaluating the let binding itself.

\subsubsection{Addition to the typing rules \cite{pierce2002ProgLang}}
\begin{equation*}
    \tag{Let}
    \infer{
        \Gamma \vdash \texttt{let } x=t_1 \texttt{ in } t_2 \colon \texttt{T}_2
    }{
        \Gamma \vdash t_1 \colon \texttt{T}_1 & \Gamma , x \colon \texttt{T}_1 \vdash t_2 \colon \texttt{T}_2
    }
\end{equation*}
To calculate the type of the let binding, calculate the type of the bound term. The bound
term will yield the same type as the used term for the binding.

\subsection{Sums and Variants}
\label{subsec:variants}

Many programs need to tackle variants. This means
that we can sum together multiple shapes of a type into a summary
type. Such variants are \textit{algebraic data types} and are often used in functional
languages for pattern matching. One can compare them vaguely to \textit{Enums} of
object-oriented languages like C\#.
This paper will use the generalized definition
of the variant to describe the principle. Thus, instead of a sum type \texttt{T + T},
we use the labeled variant type $\langle l_1 \colon \texttt{T}_1, l_2 \colon \texttt{T}_2 \rangle$.
The sum type could be compared to Haskell's ``\textit{Either a b}'' type, which can be
either type ``a'' or ``b''.

\subsubsection{Addition to the syntax \cite{pierce2002ProgLang}}
\begin{bnfgrammar}
    t ::= : terms$\colon$
    | \dots
    | <l=t> as T : tagging
    | case t of <$l_i=x_i$> $\Rightarrow t_i^{i \in 1..n}$ : case
\end{bnfgrammar}\leavevmode\newline
This syntax allows generalized labeled variants of types.

\begin{bnfgrammar}
    T ::= : types$\colon$
    | \dots
    | <$l_i \colon T_i^{i \in 1..n}$> : type of variants
\end{bnfgrammar}\leavevmode\newline

\subsubsection{Addition to the evaluation rules \cite{pierce2002ProgLang}}
\begin{equation*}
    \tag{Case variant}
    \begin{split}
        \texttt{case} (\text{<}l_j=v_j\text{>} \texttt{ as } \texttt{T}) \texttt{ of } \text{<}l_i=x_i\text{>} \Rightarrow t_i^{i \in 1..n} \\
        \rightarrow [x_j \mapsto v_j]t_j
    \end{split}
\end{equation*}
Check the variant with a \texttt{case variant} syntax and return
the given term to the right-hand side of the arrow. Replace the
x variable in the term with the ascribed type.

\begin{equation*}
    \tag{Case}
    \infer{
        \begin{split}
            \texttt{case } t_0 \texttt{ of } \text{<}l_i=x_i\text{>} \Rightarrow t_i^{i \in 1..n} \\
            \rightarrow \texttt{case } t'_0 \texttt{ of } \text{<}l_i=x_i\text{>} \Rightarrow t_i^{i \in 1..n}
        \end{split}
    }{
        t_0 \rightarrow t'_0
    }
\end{equation*}
If there is a step from $t_0$ to $t'_0$, evaluate the term
in the case clause before applying any mapping.

\begin{equation*}
    \tag{Variant}
    \infer{
        \text{<}l_i=t_i\text{>} \texttt{ as T } \rightarrow \text{<}l_i=t'_i\text{>} \texttt{ as T }
    }{
        t_i \rightarrow t'_i
    }
\end{equation*}
If there is a step from $t_i$ to $t'_i$, evaluate the term
in the variant.

\subsubsection{Addition to the typing rules \cite{pierce2002ProgLang}}
\begin{equation*}
    \tag{Variant}
    \infer{
    \Gamma \vdash \text{<}l_j=t_j\text{>} \texttt{ as } \text{<}l_i \colon T_i^{i \in 1..n}\text{>} \colon \text{<}l_i \colon T_i^{i \in 1..n}\text{>}
    }{
    \Gamma \vdash t_j \colon \texttt{T}_j
    }
\end{equation*}
When the type of $t_j$ is in the typing environment, add the labeled
variant types in the environment as well with their corresponding labels and
term types.

\begin{equation*}
    \tag{Case}
    \infer{
    \Gamma \vdash \texttt{case } t_0 \texttt{ of } \text{<}l_i=x_i\text{>} \Rightarrow t_i^{i \in 1..n} \colon \texttt{T}
    }{
    \begin{split}
        \Gamma \vdash t_0 \colon \text{<}l_i \colon \texttt{T}_i^{i \in 1..n}\text{>} \\
        \text{for each } i \text{    } \Gamma, x_i \colon \texttt{T}_i \vdash t_i \colon T
    \end{split}
    }
\end{equation*}
If $t_0$ is a variant with $i$ label (and therefore variants), the `case of' syntax
will return the specific type of the variant instead of the summary type.

With the variant types in place, our language could now type-check and interpret
the following lines of code (given in the Haskell syntax for readability):

\begin{minted}{Haskell}
    data StringOrInt = MyString String | MyInt Int
    getStringValue :: StringOrInt -> String
    getStringValue value = case value of
        MyString s -> s
        MyInt i -> show i
\end{minted}

Variants are often used to represent variable return values.
One can think of the ``Option'' type in F\# or the ``Maybe'' type
of Haskell. Both define two variants of a result, namely ``Some'' (``Just'')
or ``None'' (``Nothing''). A computation that may have a none result can return
this type constructor of the variant and can signal an empty or faulty result.
A typical case could be number parsing. When one wants to parse the string
\texttt{"12"} into a number, the result in Haskell could be \texttt{"Just 12"}, but
on the other hand, parsing \texttt{"12i"} would result in \texttt{"Nothing"}.

\subsection{Records}
\label{subsec:Records}

Since tuples have indices and must be accessed that way (as stated in the overview), we may want to
name the elements in a tuple. ``Records'' provide a way to label the entries
of a tuple and create a possibility to semantically group terms together.
One could loosely compare them with \texttt{Structs} from programming
languages like \texttt{Go}.

\subsubsection{Addition to the syntax \cite{pierce2002ProgLang}}
\begin{bnfgrammar}
    t ::= : terms$\colon$
    | \dots
    | $\{l_i = t_i^{i \in 1..n}\}$ : record
    | t.l : projection
\end{bnfgrammar}\leavevmode\newline
This syntax rule follows the same principle as for the tuple. One change to
note is that the projection is not done via an index but with a label $l$.

\begin{bnfgrammar}
    v ::= : values$\colon$
    | \dots
    | $\{l_i = v_i^{i \in 1..n}\}$ : record value
\end{bnfgrammar}\leavevmode\newline

\begin{bnfgrammar}
    T ::= : types$\colon$
    | \dots
    | $\{l_i \colon T_i^{i \in 1..n}\}$ : tuple of records
\end{bnfgrammar}\leavevmode\newline

\subsubsection{Addition to the evaluation rules \cite{pierce2002ProgLang}}
\begin{equation*}
    \tag{Record projection}
    \{l_i = v_i^{i \in 1..n}\} . l_j \rightarrow v_j
\end{equation*}
When the projection with label $j$ is applied to a record with
$i$ values, return the value with the label $j$.

\begin{equation*}
    \tag{Projection}
    \infer{
        t_1 . l \rightarrow t'_1 . l
    }{
        t_1 \rightarrow t'_1
    }
\end{equation*}
If there is a step from $t_1$ to $t'_1$, evaluate the step
in the syntax before executing the projection.

\begin{equation*}
    \tag{Record}
    \infer{
        \begin{split}
            \{l_i = v_i^{i \in 1..j-1}, l_j = t_j, l_k = t_k^{k \in j+1..n}\} \\
            \rightarrow \{l_i = v_i^{i \in 1..j-1}, l_j = t'_j, l_k = t_k^{k \in j+1..n}\}
        \end{split}
    }{
        t_j \rightarrow t'_j
    }
\end{equation*}
If there is a step from $t_j$ to $t'_j$, evaluate the leftmost
term $l_j = t_j$ to $l_j = t'_j$ that is not a value. This enforces
the same evaluation rules on records as the above rules did on tuples.
In other terms:
$\{\text{foo}=t_1, \text{bar}=t_2\} \mapsto \{\text{foo}=v_1, \text{bar}=t_2\} \mapsto \{\text{foo}=v_1, \text{bar}=v_2\}$.

\subsubsection{Addition to the typing rules \cite{pierce2002ProgLang}}
\begin{equation*}
    \tag{Record}
    \infer{
    \Gamma \vdash \{l_i = t_i^{i \in 1..n}\} \colon \{l_i \colon \texttt{T}_i^{i \in 1..n}\}
    }{
    \text{for each } i & \Gamma \vdash t_i \colon \texttt{T}_i
    }
\end{equation*}
For each element in the record with label $l$, we calculate the type
and add the whole record to the typing context $\Gamma$ in the form
$\{l_1 \colon T_1, l_2 \colon T_2, \dots\}$.

\begin{equation*}
    \tag{Projection}
    \infer{
    \Gamma \vdash t_i . l_j \colon \texttt{T}_j
    }{
    \Gamma \vdash t_1 \colon \{l_i \colon \texttt{T}_i^{i \in 1..n}\}
    }
\end{equation*}
If the term $t_1$ is a record type with $i$ entries, the projection
$t_1 . l_j$ will yield an element of type $\texttt{T}_j$ at
the position of label $l_j$.
