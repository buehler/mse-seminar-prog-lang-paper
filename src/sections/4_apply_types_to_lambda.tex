\section{$\lambda$-Calculus, Simple Types and Extensions}

When we apply ``simple types'' to the purely untyped
$\lambda$-Calculus, the result is the ``pure simply typed $\lambda$-Calculus''.
This could count as a programming language since we can perform
all basic operations a computation needs. We're also able to analyze
the code and calculate the types needed for functions and variables
and therefore can categorize our programs into meaningful and
meaningless ones. To make the syntax more useful however, we can extend
our simple types with ``simple extensions'' which do not include any
form of polymorphism. Those extensions make our language and the type
checker more useful and able to perform operations.

The following sections will explain such extensions and how they are constructed.
For the further reading, the rules of the ``pure simply typed $\lambda$-Calculus''
are stated again \cite{pierce2002ProgLang}:

~\\
\textbf{Syntax}
\begin{bnfgrammar}
    t ::= : terms$\colon$
    | x : variable
    | $\lambda$x$\colon$T . t : abstraction
    | t t : application
\end{bnfgrammar}\leavevmode\newline

\begin{bnfgrammar}
    v ::= : values$\colon$
    | $\lambda$x$\colon$T . t : abstraction value
\end{bnfgrammar}\leavevmode\newline

\begin{bnfgrammar}
    T ::= : types$\colon$
    | T$\rightarrow$T : type of functions
\end{bnfgrammar}\leavevmode\newline

\begin{bnfgrammar}
    $\Gamma$ ::= : contexts$\colon$
    | $\varnothing$ : empty context
    | $\Gamma$, x$\colon$T : term variable binding
\end{bnfgrammar}\leavevmode\newline

\textbf{Evaluation}
\begin{equation*}
    \tag{Application 1}
    \infer{
        t_1 t_2 \rightarrow t'_1 t_2
    }{
        t_1 \rightarrow t'_1
    }
\end{equation*}
``If there exists an evaluation step from $t_1$ to $t'_1$ take this step
prior to the evaluation step of $t_2$''. This forces the program, to first
evaluate all terms of $t_1$ until the next rule hits.

\begin{equation*}
    \tag{Application 2}
    \infer{
        v_1 t_2 \rightarrow v_1 t'_2
    }{
        t_2 \rightarrow t'_2
    }
\end{equation*}
``If there exists an evaluation step from $t_2$ to $t'_2$ and the left
side of the application is already a value, take this step''. This defines
that when the lefthand side of the application is reduced to a value, evaluate
the righthand side.

\begin{equation*}
    \tag{Application Abstraction}
    (\lambda x \colon T_{11} . t_{12}) v_2 \rightarrow [x \mapsto v_2] t_{12}
\end{equation*}
``Replace the variable $x$ with the value $v_2$ in the term $t_{12}$''. This represents
the effective computation or application of the value to a term.

~\\
\textbf{Typing}
\begin{equation*}
    \tag{Variable}
    \infer{
        \Gamma \vdash x \colon \texttt{T}
    }{
        x \colon \texttt{T} \in \Gamma
    }
\end{equation*}

\begin{equation*}
    \tag{Application}
    \infer{
        \Gamma \vdash t_1 t_2 \colon \texttt{T}_{12}
    }{
        \Gamma \vdash t_1 \colon \texttt{T}_{11} \rightarrow \texttt{T}_{12}
        &
        \Gamma \vdash t_2 \colon \texttt{T}_{11}
    }
\end{equation*}

\begin{equation*}
    \tag{Abstraction}
    \infer{
        \Gamma \vdash \lambda x \colon \texttt{T}_1 . t_2 \colon \texttt{T}_1 \rightarrow \texttt{T}_2
    }{
        \Gamma,x \colon \texttt{T}_1 \vdash t_2 \colon \texttt{T}_2
    }
\end{equation*}

This in combination with base and function types will be the
cornerstone of the ``simple extensions'' which we will see in the
next sections. Those sections will introduce new elements to the given
categories (``syntax'', ``evaluation'', ``typing'').

\subsection{Unit Type}

The \texttt{unit} type represents a useful type often found in functional
programming languages like Haskell or F\#. It is used to ``throw away'' a
computation result and combine multiple computations together \cite{pierce2002ProgLang}.
In Haskell this can be used in the \texttt{main} function to glue
several functions together that contain side effects. It can be viewed as
the ``\texttt{void}'' type in C\# or Java \cite{pierce2002ProgLang}.

Addition to the syntax \cite{pierce2002ProgLang}:

\begin{bnfgrammar}
    t ::= : terms$\colon$
    | \dots
    | unit : constant unit
\end{bnfgrammar}\leavevmode\newline

\begin{bnfgrammar}
    v ::= : values$\colon$
    | \dots
    | unit : constant unit
\end{bnfgrammar}\leavevmode\newline

\begin{bnfgrammar}
    T ::= : types$\colon$
    | \dots
    | Unit : unit type
\end{bnfgrammar}\leavevmode\newline

Addition to the typing rules \cite{pierce2002ProgLang}:
\begin{equation*}
    \tag{Unit}
    \Gamma \vdash \texttt{unit} \colon \texttt{Unit}
\end{equation*}\leavevmode\newline

Added derived form \cite{pierce2002ProgLang}:
\begin{equation*}
    t_1 ; t_2 \quad \defeq \quad (\lambda x \colon \texttt{Unit} . t_2) t_1 \text{ where } x \notin FV(t_2)
\end{equation*}
``The function is applied to the term $t_1$ where the input variable $x$ is not
part of the ``free variables''\footnote{Free variables are not bound variables in the term.}
(FV) of the term $t_2$''.

~\\
Addition to the evaluation rules \cite{pierce2002ProgLang}:
\begin{equation*}
    \tag{Sequence}
    \infer{
        t_1 ; t_2 \rightarrow t'_1 ; t_2
    }{
        t_1 \rightarrow t'_1
    }
\end{equation*}
``If there is a sequence (noted by ;) and there is a step from $t_1$ to
$t'_1$, evaluate the term $t_1$''.

\begin{equation*}
    \tag{Sequence Next}
    \texttt{unit} ; t_2 \rightarrow t_2
\end{equation*}
``If the lefthand side of a sequence is reduced to \texttt{unit},
return the result of $t_2$''.

~\\
Addition to the typing rules \cite{pierce2002ProgLang}:
\begin{equation*}
    \tag{Sequence}
    \infer{
        \Gamma \vdash t_1 ; t_2 \colon \texttt{T}_2
    }{
        \Gamma \vdash t_1 \colon \texttt{Unit} & \Gamma \vdash t_2 \colon \texttt{T}_2
    }
\end{equation*}
``If $t_1$ is of type \texttt{Unit} and $t_2$ has a type $\texttt{T}_2$ then
the resulting type of the sequence will be $\texttt{T}_2$''.

\subsection{Ascription}

A very handy tool for our simple programming language is the usage of ascription.
It is often used for ``documentation'' purposes \cite{pierce2002ProgLang}. It defines
a way to substitute long type names with shorter ones. An example for such a substitute
in the Haskell language would be: ``\texttt{type MyType = Double -> Double -> [Char]}''
which defines the type \texttt{MyType} as a function that takes a double and a double
and returns an array of characters.

Addition to the syntax \cite{pierce2002ProgLang}:
\begin{bnfgrammar}
    t ::= : terms$\colon$
    | \dots
    | t as T : ascription
\end{bnfgrammar}\leavevmode\newline

Addition to the evaluation rules \cite{pierce2002ProgLang}:
\begin{equation*}
    \tag{Ascribe Value}
    v_1 \text{ as } \texttt{T} \rightarrow v_1
\end{equation*}
``The term $v_1 \text{ as } \texttt{T}$ returns $v_1$''.

\begin{equation*}
    \tag{Ascription Evaluation}
    \infer{
        t_1 \text{ as } \texttt{T} \rightarrow t'_1 \text{ as } \texttt{T}
    }{
        t_1 \rightarrow t'_1
    }
\end{equation*}
``If there is a step from $t_1$ to $t'_1$, evaluate the step
in the syntax''.

~\\
Addition to the typing rules \cite{pierce2002ProgLang}:
\begin{equation*}
    \tag{Ascribe}
    \infer{
        \Gamma \vdash t_1 \text{ as } \texttt{T} \colon \texttt{T}
    }{
        \Gamma \vdash t_1 \colon \texttt{T}
    }
\end{equation*}
``If $t_1$ is assumed with the type \texttt{T} in the context,
the term $t_1 \text{ as } \texttt{T}$ will yield a type \texttt{T}''.

\subsection{Let Bindings}

Let bindings are a useful tool to avoid repetition in complex expressions.
They are found in Haskell as well:

\begin{minted}{Haskell}
    add x y =
        let result = x + y
        in
            result
\end{minted}

Addition to the syntax \cite{pierce2002ProgLang}:
\begin{bnfgrammar}
    t ::= : terms$\colon$
    | \dots
    | let x=t in t : let binding
\end{bnfgrammar}\leavevmode\newline

Addition to the evaluation rules \cite{pierce2002ProgLang}:
\begin{equation*}
    \tag{Let-Bind Value}
    \texttt{let } x=v_1 \texttt{ in } t_2 \rightarrow [x \mapsto v_1] t_2
\end{equation*}
``In the given abstraction $t_2$ replace all occurencies of $x$ with $v_1$''.

\begin{equation*}
    \tag{Let}
    \infer{
        \texttt{let } x=t_1 \texttt{ in } t_2 \rightarrow \texttt{let } x=t'_1 \texttt{ in } t_2
    }{
        t_1 \rightarrow t'_1
    }
\end{equation*}
``If there is a step from $t_1$ to $t'_1$, evaluate the step
in the syntax before evaluating the let binding itself''.

~\\
Addition to the typing rules \cite{pierce2002ProgLang}:
\begin{equation*}
    \tag{Let}
    \infer{
        \Gamma \vdash \texttt{let } x=t_1 \texttt{ in } t_2 \colon \texttt{T}_2
    }{
        \Gamma \vdash t_1 \colon \texttt{T}_1 & \Gamma , x \colon \texttt{T}_1 \vdash t_2 \colon \texttt{T}_2
    }
\end{equation*}
``To calculate the type of the let binding, calculate the type of the bound term. The bound
term will yield the same type as the used term for the binding''.

\subsection{Pairs and Tuples}

Until now, the additions were minor and added some syntactic sugar to the language of
the simple typed $\lambda$-Calculus. The following simple extensions will enrich
the language with features that are often found in programming languages.

Pairs - and their more general counterpart Tuples - are a construct to group
values and terms together. Pairs are product types of exactly two values
and therefore have slightly different evaluation rules. Tuples are the general way
of pairs and therefore only the rules of tuples will be explained since they include
the rules of pairs as well.

Addition to the syntax \cite{pierce2002ProgLang}:
\begin{bnfgrammar}
    t ::= : terms$\colon$
    | \dots
    | $\{t_i^{i \in 1..n}\}$ : tuple
    | t.i : projection
\end{bnfgrammar}\leavevmode\newline
$\{t_i^{i \in 1..n}\}$ means that for example with $i=3$ we have a tuple
of three elements. A tuple with $i=2$ would be a pair.
$\{t_i^{i \in 1..n}\} \text{ with } i=3 \mapsto \{t_1, t_2, t_3\}$.
The projection is needed to access the elements in a tuple at the given index.

\begin{bnfgrammar}
    v ::= : values$\colon$
    | \dots
    | $\{v_i^{i \in 1..n}\}$ : tuple value
\end{bnfgrammar}\leavevmode\newline

\begin{bnfgrammar}
    T ::= : types$\colon$
    | \dots
    | $\{T_i^{i \in 1..n}\}$ : tuple type
\end{bnfgrammar}\leavevmode\newline

Addition to the evaluation rules \cite{pierce2002ProgLang}:
\begin{equation*}
    \tag{Tuple projection}
    \{v_i^{i \in 1..n}\} . j \rightarrow v_j
\end{equation*}
``When the projection with index $j$ is applied to a tuple with
$i$ values, return the value at index $j$''.

\begin{equation*}
    \tag{Projection}
    \infer{
        t_1 . i \rightarrow t'_1 . i
    }{
        t_1 \rightarrow t'_1
    }
\end{equation*}
``If there is a step from $t_1$ to $t'_1$, evaluate the step
in the syntax before executing the projection''.

\begin{equation*}
    \tag{Tuple}
    \infer{
    \{v_i^{i \in 1..j-1}, t_j, t_k^{k \in j+1..n}\} \rightarrow \{v_i^{i \in 1..j-1}, t'_j, t_k^{k \in j+1..n}\}
    }{
    t_j \rightarrow t'_j
    }
\end{equation*}
``If there is a step from $t_j$ to $t'_j$, evaluate the leftmost
term $t_j$ to $t'_j$ that is not a value''. This forces the tuple to
be fully evaluated before any projections can be executed on the tuple.
Also, it enforces the evaluation direction for the tuple from left to write.
In other terms: $\{t_1, t_2\} \mapsto \{v_1, t_2\} \mapsto \{v_1, v_2\}$.

~\\
Addition to the typing rules \cite{pierce2002ProgLang}:
\begin{equation*}
    \tag{Tuple}
    \infer{
    \Gamma \vdash \{t_i^{i \in 1..n}\} \colon \{\texttt{T}_i^{i \in 1..n}\}
    }{
    \text{for each } i & \Gamma \vdash t_i \colon \texttt{T}_i
    }
\end{equation*}
``For each element in the tuple with index $i$, we calculate the type
and add the whole tuple to the typing context $\Gamma$ in the form $\{T_1, T_2, \dots\}$''.

\begin{equation*}
    \tag{Projection}
    \infer{
    \Gamma \vdash t_i . j \colon \texttt{T}_j
    }{
    \Gamma \vdash t_1 \colon \{\texttt{T}_i^{i \in 1..n}\}
    }
\end{equation*}
``If the term $t_1$ is a tuple type with $i$ entries, the projection
$t_1 . j$ will yield an element of the type $\texttt{T}_j$''.

\subsection{Records}

Since tuples have indices and must be accessed that way, we may want to
name the elements in a tuple. ``Records'' provide a way to label entries
of a tuple and create a possibility to semantically group terms together.
One could loosely compare them with \texttt{Structs} from programming
languages like \texttt{GoLang}.

Addition to the syntax \cite{pierce2002ProgLang}:
\begin{bnfgrammar}
    t ::= : terms$\colon$
    | \dots
    | $\{l_i = t_i^{i \in 1..n}\}$ : record
    | t.l : projection
\end{bnfgrammar}\leavevmode\newline
This syntax rule follows the same principle as for the tuple. One change to
note is, that the projection is not done via an index but with a label $l$.

\begin{bnfgrammar}
    v ::= : values$\colon$
    | \dots
    | $\{l_i = v_i^{i \in 1..n}\}$ : record value
\end{bnfgrammar}\leavevmode\newline

\begin{bnfgrammar}
    T ::= : types$\colon$
    | \dots
    | $\{l_i \colon T_i^{i \in 1..n}\}$ : tuple of records
\end{bnfgrammar}\leavevmode\newline

Addition to the evaluation rules \cite{pierce2002ProgLang}:
\begin{equation*}
    \tag{Record projection}
    \{l_i = v_i^{i \in 1..n}\} . l_j \rightarrow v_j
\end{equation*}
``When the projection with label $j$ is applied to a record with
$i$ values, return the value with the label $j$''.

\begin{equation*}
    \tag{Projection}
    \infer{
        t_1 . l \rightarrow t'_1 . l
    }{
        t_1 \rightarrow t'_1
    }
\end{equation*}
``If there is a step from $t_1$ to $t'_1$, evaluate the step
in the syntax before executing the projection''.

\begin{equation*}
    \tag{Record}
    \infer{
    \{l_i = v_i^{i \in 1..j-1}, l_j = t_j, l_k = t_k^{k \in j+1..n}\}
    \rightarrow \{l_i = v_i^{i \in 1..j-1}, l_j = t'_j, l_k = t_k^{k \in j+1..n}\}
    }{
    t_j \rightarrow t'_j
    }
\end{equation*}
``If there is a step from $t_j$ to $t'_j$, evaluate the leftmost
term $l_j = t_j$ to $l_j = t'_j$ that is not a value''. This inforces
the same evaluation rules on records as the above rules did on tuples.
In other terms:
$\{\text{foo}=t_1, \text{bar}=t_2\} \mapsto \{\text{foo}=v_1, \text{bar}=t_2\} \mapsto \{\text{foo}=v_1, \text{bar}=v_2\}$.

~\\
Addition to the typing rules \cite{pierce2002ProgLang}:
\begin{equation*}
    \tag{Record}
    \infer{
    \Gamma \vdash \{l_i = t_i^{i \in 1..n}\} \colon \{l_i \colon \texttt{T}_i^{i \in 1..n}\}
    }{
    \text{for each } i & \Gamma \vdash t_i \colon \texttt{T}_i
    }
\end{equation*}
``For each element in the record with label $l$, we calculate the type
and add the whole record to the typing context $\Gamma$ in the form
$\{l_1 \colon T_1, l_2 \colon T_2, \dots\}$''.

\begin{equation*}
    \tag{Projection}
    \infer{
    \Gamma \vdash t_i . l_j \colon \texttt{T}_j
    }{
    \Gamma \vdash t_1 \colon \{l_i \colon \texttt{T}_i^{i \in 1..n}\}
    }
\end{equation*}
``If the term $t_1$ is a record type with $i$ entries, the projection
$t_1 . l_j$ will yield an element of the type $\texttt{T}_j$ at
the position of label $l_j$''.

\subsection{Sums and Variants}

Many programs need to tackle variant types. This means
that we can sum together multiple shapes of a type into a summary
type. Such variants are algebraic data types and are often used in functional
languages with pattern matching. This paper will use the generalized definition
of the variant to describe the principle. So instead of a sum type \texttt{T + T}
we use the labeled variant type $\langle l_1 \colon \texttt{T}_1, l_2 \colon \texttt{T}_2 \rangle$.



With the variant types in place, our language could now type-check and interpret
the following lines of code (given in the Haskell syntax for readability):

\begin{minted}{Haskell}
    type Weekend = Saturday | Sunday
    getName day = case day of
        Saturday -> "Saturday"
        Sunday -> "Sunday"
\end{minted}

\subsection{General Recursion}
\subsection{Lists}
\subsection{References}
\subsection{Exceptions}
\subsection{Summary}

With those possibilities, a variety of concepts can be created.
While there exists the concept of ``exceptions'' for example, it does
not support subtyping of any kind. Polymorphism is part of a higher
version of a type system, for example System F.

\todo{reference to marcs paper about systemF}

The given base of the ``pure simple typed $\lambda$-Calculus'' and
the simple extensions above now give us a language that can successfully compile
and compute the following lines of (TypeScript-ish) code.

\todo{Code}

Since the $\lambda$-Calculus is the mathematical foundation of several functional
programming languages, the features (without polymorphism) can be translated
into them. Let us review the stated features in a pure functional language
like Haskell:

\todo{Code}
