\subsection{Records}

Since tuples have indices and must be accessed that way, we may want to
name the elements in a tuple. ``Records'' provide a way to label entries
of a tuple and create a possibility to semantically group terms together.
One could loosely compare them with \texttt{Structs} from programming
languages like \texttt{GoLang}.

\subsubsection{Addition to the syntax \cite{pierce2002ProgLang}}
\begin{bnfgrammar}
    t ::= : terms$\colon$
    | \dots
    | $\{l_i = t_i^{i \in 1..n}\}$ : record
    | t.l : projection
\end{bnfgrammar}\leavevmode\newline
This syntax rule follows the same principle as for the tuple. One change to
note is, that the projection is not done via an index but with a label $l$.

\begin{bnfgrammar}
    v ::= : values$\colon$
    | \dots
    | $\{l_i = v_i^{i \in 1..n}\}$ : record value
\end{bnfgrammar}\leavevmode\newline

\begin{bnfgrammar}
    T ::= : types$\colon$
    | \dots
    | $\{l_i \colon T_i^{i \in 1..n}\}$ : tuple of records
\end{bnfgrammar}\leavevmode\newline

\subsubsection{Addition to the evaluation rules \cite{pierce2002ProgLang}}
\begin{equation*}
    \tag{Record projection}
    \{l_i = v_i^{i \in 1..n}\} . l_j \rightarrow v_j
\end{equation*}
``When the projection with label $j$ is applied to a record with
$i$ values, return the value with the label $j$''.

\begin{equation*}
    \tag{Projection}
    \infer{
        t_1 . l \rightarrow t'_1 . l
    }{
        t_1 \rightarrow t'_1
    }
\end{equation*}
``If there is a step from $t_1$ to $t'_1$, evaluate the step
in the syntax before executing the projection''.

\begin{equation*}
    \tag{Record}
    \infer{
        \begin{split}
            \{l_i = v_i^{i \in 1..j-1}, l_j = t_j, l_k = t_k^{k \in j+1..n}\} \\
            \rightarrow \{l_i = v_i^{i \in 1..j-1}, l_j = t'_j, l_k = t_k^{k \in j+1..n}\}
        \end{split}
    }{
        t_j \rightarrow t'_j
    }
\end{equation*}
``If there is a step from $t_j$ to $t'_j$, evaluate the leftmost
term $l_j = t_j$ to $l_j = t'_j$ that is not a value''. This inforces
the same evaluation rules on records as the above rules did on tuples.
In other terms:
$\{\text{foo}=t_1, \text{bar}=t_2\} \mapsto \{\text{foo}=v_1, \text{bar}=t_2\} \mapsto \{\text{foo}=v_1, \text{bar}=v_2\}$.

\subsubsection{Addition to the typing rules \cite{pierce2002ProgLang}}
\begin{equation*}
    \tag{Record}
    \infer{
    \Gamma \vdash \{l_i = t_i^{i \in 1..n}\} \colon \{l_i \colon \texttt{T}_i^{i \in 1..n}\}
    }{
    \text{for each } i & \Gamma \vdash t_i \colon \texttt{T}_i
    }
\end{equation*}
``For each element in the record with label $l$, we calculate the type
and add the whole record to the typing context $\Gamma$ in the form
$\{l_1 \colon T_1, l_2 \colon T_2, \dots\}$''.

\begin{equation*}
    \tag{Projection}
    \infer{
    \Gamma \vdash t_i . l_j \colon \texttt{T}_j
    }{
    \Gamma \vdash t_1 \colon \{l_i \colon \texttt{T}_i^{i \in 1..n}\}
    }
\end{equation*}
``If the term $t_1$ is a record type with $i$ entries, the projection
$t_1 . l_j$ will yield an element of the type $\texttt{T}_j$ at
the position of label $l_j$''.
