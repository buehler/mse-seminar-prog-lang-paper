\subsection{Ascription}

A very handy tool for our simple programming language is the usage of ascription.
It is often used for ``documentation'' purposes \cite{pierce2002ProgLang}. It defines
a way to substitute long type names with shorter ones. An example for such a substitute
in the Haskell language would be: ``\texttt{type MyType = Double -> Double -> [Char]}''
which defines the type \texttt{MyType} as a function that takes a double and a double
and returns an array of characters.

\subsubsection{Addition to the syntax \cite{pierce2002ProgLang}}
\begin{bnfgrammar}
    t ::= : terms$\colon$
    | \dots
    | t as T : ascription
\end{bnfgrammar}\leavevmode\newline

\subsubsection{Addition to the evaluation rules \cite{pierce2002ProgLang}}
\begin{equation*}
    \tag{Ascribe Value}
    v_1 \text{ as } \texttt{T} \rightarrow v_1
\end{equation*}
``The term $v_1 \text{ as } \texttt{T}$ returns $v_1$''.

\begin{equation*}
    \tag{Ascription Evaluation}
    \infer{
        t_1 \text{ as } \texttt{T} \rightarrow t'_1 \text{ as } \texttt{T}
    }{
        t_1 \rightarrow t'_1
    }
\end{equation*}
``If there is a step from $t_1$ to $t'_1$, evaluate the step
in the syntax''.

\subsubsection{Addition to the typing rules \cite{pierce2002ProgLang}}
\begin{equation*}
    \tag{Ascribe}
    \infer{
        \Gamma \vdash t_1 \text{ as } \texttt{T} \colon \texttt{T}
    }{
        \Gamma \vdash t_1 \colon \texttt{T}
    }
\end{equation*}
``If $t_1$ is assumed with the type \texttt{T} in the context,
the term $t_1 \text{ as } \texttt{T}$ will yield a type \texttt{T}''.
