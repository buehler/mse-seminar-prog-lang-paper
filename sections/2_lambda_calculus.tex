\section{Lambda-Calculus ($\lambda$-Calculus)}

A computer program can be described in various
ways. One very famous variant is the ``turing machine''
which was defined in a journal \cite{aturingMachine} by
A. M. Turing in the year 1937. The Turing machine
is fed with instructions and contains a ``memory'' band
to write down results for further computation.

Another famous - but more abstract - method to describe
a computation is the $\lambda$-Calculus. This system was
specified by Alonzo Church \cite{churchLambda}. It is a mathematical
model of computation that only contains three rules
of operation \cite{pierce2002ProgLang}.
Those three operations can be viewed in the following ``grammar'':

\begin{bnfgrammar}
    t ::= : terms$\colon$
    | x : variable
    | $\lambda$x.t : abstraction
    | t t : application
\end{bnfgrammar}

The untyped $\lambda$-Calculus is Turing complete,
which means it can compute \textit{any} program and
therefore can run infinitely.
In an untyped $\lambda$ system, it is possible to search
for the successor of ``true'', which
requires the argument to be a number and therefore
results in a stuck state.

Since it is not desirable for computer programs to run to infinity,
there has to be a way to split up computer programs into two categories:
The ``useful'' and ``useless'' ones. Any program that will run forever
or that will be stuck in an error state counts towards the ``useless'' ones.
Other programs that have valid inputs and outputs will be counted towards ``useful''
programs.

For the further progress of this paper, it is necessary to recall
the grammar for arithmetic expressions of the untyped calculus \cite{pierce2002ProgLang}:

\subsection{Terms}

\begin{bnfgrammar}
    t ::= : terms$\colon$
    | true : constant true
    | false : constant false
    | if t then t else t : conditional
    | 0 : constant zero
    | succ t : successor
    | pred t : predecessor
    | iszero t : zero test
\end{bnfgrammar}

\subsection{Values}

\begin{bnfgrammar}
    v ::= : values$\colon$
    | true : true value
    | false : false value
    | nv : numeric value
\end{bnfgrammar}

\subsection{Numeric Values}

\begin{bnfgrammar}
    nv ::= : numeric values$\colon$
    | 0 : zero value
    | succ nv : successor value
\end{bnfgrammar}
